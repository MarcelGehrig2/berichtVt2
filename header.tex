\documentclass[
  10pt,               % 120% Schriftgrösse
  oneside,            % einsitiger Druck
  a4paper,            % A4
  titlepage,          % inklusive Titelpage
  pointlessnumbers,	  % Kein Punkt hinter der Kapitelnummerierung
  halfparskip,        % Europäischer Satz mit abstand zwischen Absätzen
  pdftex,             % Direkt ins Pdf übersetzen, keine Kapitel
  liststotoc,         % Inhaltsverzeichnis inkl. Abbildungsverzeichnis
  bibtotoc]{scrreprt} % Inhaltsverzeichnis inkl. Literaturverzeichnis

%%%%%%%%%%%%%%%%%%%%%%%%%
% Allgemeine Packete    %
%%%%%%%%%%%%%%%%%%%%%%%%%
\usepackage{pifont}
\newcommand{\cmark}{\ding{51}}%
\newcommand{\xmark}{\ding{55}}%

% background color for text
\usepackage{xcolor}
\usepackage{framed}
%\definecolor{shadecolor}{RGB}{180,180,180}
\definecolor{shadecolor}{RGB}{220,220,220}

\usepackage{enumerate}

% Neue Rechtschreibung
\usepackage[german, ngerman]{babel}

% UTF 8
\usepackage[utf8]{inputenc}
\usepackage{units}
\usepackage{multirow}

% Ausgabefonts
\usepackage[T1]{fontenc}

% Euro Symbol (\texteuro}
\usepackage{textcomp}

% für die Gesamtseitenzahl
\usepackage{totpages}

% Paket für Farben
\usepackage{color}

% Bilder
%\usepackage{graphicx}

% Fliesstext um Bilder
\usepackage{wrapfig}

% Tabellen mit definierter Breite und zentriert
\usepackage{array}

\newcolumntype{x}[1]{%
>{\centering\hspace{0pt}}p{#1}}%

% Glossar alt
%\usepackage[style=super, header=none, border=none, number=none, cols=2, toc=true]{glossary}
%\makeglossary

% Glossar neu
%\usepackage{glossaries}
%\makeglossaries
%\printglossaries

% mehrere Befehle für die Tabellen
\usepackage{booktabs}

%Packet für absolut Positionierte Textboxen
\usepackage[absolute]{textpos} %showboxes zum besseren positionieren

% Packet für Boxen
%\usepackage{framed}

% Initialen
%\usepackage{lettrine}
%\DeclareFixedFont{\Yinit}{U}{yinit}{m}{n}{16}

% checkmarks
\usepackage{tikz}
\def\checkmark{\tikz\fill[scale=0.4](0,.35) -- (.25,0) -- (1,.7) -- (.25,.15) -- cycle;} 

%%%%%%%%%%%%%%%%%%%%%%%%%
% Seiteneinstellungen   %
%%%%%%%%%%%%%%%%%%%%%%%%%
\usepackage[
  portrait,
%   twoside,
  left=40mm,
  top=20mm,
  textwidth=145mm,
  textheight=252mm,
  %head=15mm,
  head=15mm,
  headsep=5mm,
  foot=8mm
]{geometry}

% Text muss nicht umbedingt bis zur Fusszeile reichen
\raggedbottom

%\usepackage[cam,axes,a3,center]{crop}
%\usepackage[frame,a4,center]{crop}

%%%%%%%%%%%%%%%%%%%%%%%%%
% Schriftart            %
%%%%%%%%%%%%%%%%%%%%%%%%%

%intent:
%USE: \-\hspace{2cm}

%\usepackage{courier}  % Courier für Code verwenden
%\usepackage{helvet}   % Helvetica
% Helvetic als Font verwenden
%\renewcommand{\familydefault}{\sfdefault}


%%%%%%%%%%%%%%%%%%%%%%%%%
% Mathematik Packete    %
%%%%%%%%%%%%%%%%%%%%%%%%%

% Verbesserte Mathematik Satz
\usepackage{amsmath}

% Zahlenmenngen in der Mathematik
\usepackage{amssymb}

% Times New Roman für Mathematik
\usepackage{mathptmx}


%%%%%%%%%%%%%%%%%%%%%%%%%
% Inhaltsverzeichnis    %
%%%%%%%%%%%%%%%%%%%%%%%%%
% Punkte im Inhaltsverzeichnis bei Sections
\usepackage{tocloft}
\renewcommand{\cftsecdotsep}{4.5}

%%%%%%%%%%%%%%%%%%%%%%%%%
% Literaturverzeichnis  %
%%%%%%%%%%%%%%%%%%%%%%%%%

%\usepackage[ngerman,ref]{backref}
%\renewcommand{\backrefpagesname}{Zitiert auf S.}
%renewcommand{\refname}{Quellenverzeichnis}

%%%%%%%%%%%%%%%%%%%%%%%%%
% Anhang			    %
%%%%%%%%%%%%%%%%%%%%%%%%%

\newcommand{\anhang}[1]{
	%\setsection{1}
    \setcounter{page}{1}
	\input{#1}
	\newpage
}

%%%%%%%%%%%%%%%%%%%%%%%%%
% Source Code Packete   %
%%%%%%%%%%%%%%%%%%%%%%%%%

% Codesegmente
\usepackage{listings}
\definecolor{darkblue}{rgb}{0,0,0.6}
\definecolor{darkred}{rgb}{0.6,0,0}
\definecolor{darkgreen}{rgb}{0,0.6,0}
\definecolor{red}{rgb}{0.98,0,0}
\lstloadlanguages{C,C++} % entsprechende Sprachen werden hier schon mal geladen, damit das Übersetzen schneller geht
\lstset{%
  language=C,
  basicstyle=\footnotesize\ttfamily,
  commentstyle=\itshape\color{darkgreen},
  keywordstyle=\bfseries\color{darkblue},
  stringstyle=\color{darkred},
  showspaces=false,
  columns=fixed,
  numbers=left,
  frame=none,
  numberstyle=\tiny,
  breaklines=true,
  showstringspaces=false,
  xleftmargin=1cm,
  tabsize=2
}%




%%%%%%%%%%%%%%%%%%%%%%%%%
% Makros                %
%%%%%%%%%%%%%%%%%%%%%%%%%

% Makros
\newcommand{\redbox}[1]
{\vspace*{2mm} \\ \fboxsep5mm \fboxrule0.5mm \fcolorbox{red}{white}{\parbox{1.5cm}{
\includegraphics[width=1cm]{Achtung.pdf}}\parbox{13cm}{\textcolor{black}{#1}
}}\vspace*{2mm}\\}

\newcommand{\redboxtitle}[2]
{\vspace*{2mm} \\ \textcolor{red}{\bfseries #1\smallskip} \\ \fboxsep5mm \fboxrule0.5mm \fcolorbox{red}{white}{\parbox{1.5cm}{
\includegraphics[width=1cm]{Achtung.pdf}}\parbox{13cm}{\textcolor{black}{#2}
}}\vspace*{2mm}\\}

% Fuer die Minipage um eine Tabelle und Abbildung nebeneinander Setzen
\makeatletter
\newcommand{\setcaptype}[1]{\renewcommand{\@captype}{#1}}
\makeatother


%%%%%%%%%%%%%%%%%%%%%%%%%
% Kopf und Fusszeile    %
%%%%%%%%%%%%%%%%%%%%%%%%%
% 7. Kopf und Fusszeilen
\usepackage{totpages}               % Paket für die Gesamtseitenzahl
\usepackage{scrpage2}
\pagestyle{scrheadings}
\renewcommand*{\chapterpagestyle}{scrheadings}
\clearscrheadfoot
%\setkomafont{pagefoot}{\normalfont\sffamily}
\automark[]{chapter}
%\ihead{NTB-Robotersteuerung}
\ihead{VT2: Eine ROS Anbindung für EEROS}
\ohead{\headmark}
\setheadsepline{0.5pt}[\color{HeadBlue}] 
%\ifoot{Benno Jung • Michael Krähenbühl • Martin Züger}
\cfoot{- \thepage{} -}
%\setfootsepline{0.5pt}
%%Kopf- und Fusszeile
\definecolor{HeadBlue}{rgb}{0,0.41,0.71}
\definecolor{FootBlack}{rgb}{0,0,0}
\addtokomafont{pagehead}{\color{HeadBlue}} 
\addtokomafont{pagefoot}{\color{FootBlack}}
%\usepackage{eso-pic}
%\usepackage{scrpage2}
%\usepackage{extramarks}
\usepackage[final]{pdfpages}
%%\pagestyle{scrheadings}
%%\clearscrheadfoot
%% Schriftart auf die Normale Schrift setzen
%%\setkomafont{pagefoot}{\normalfont\sffamily}
%%\automark[]{section} % \leftmark zeigt die aktuelle Section an, \rightmark bleibt leer
%
%%Kopfzeile
%%\ihead{NTB}
%%\ohead{\leftmark}
%%\lehead{\hspace*{1cm}\begin{large}\leftmark\end{large} \AddToShipoutPicture*{%
%%  \AtPageUpperLeft{\put(0,\LenToUnit{-9.4mm}){%
%%    \colorbox{HeadBlue}{\parbox[t][6mm]{1.8cm}{\ }}%
%%    \colorbox{black}{\parbox[t][6mm]{0.1cm}{\ }}}}}}
%%\rohead{\begin{large}\leftmark\end{large}\hspace*{1cm} \AddToShipoutPicture*{%
%%  \AtPageUpperLeft{\put(\LenToUnit{\paperwidth},\LenToUnit{-9.4mm}){\kern-2.3cm%
%%    \colorbox{black}{\parbox[t][6mm]{0.1cm}{\ }}%
%%    \colorbox{HeadBlue}{\parbox[t][6mm]{1.8cm}{\ }}}}}}
%%\rehead{\begin{large}NTB\end{large}}
%%\lohead{\begin{large}NTB\end{large}}
%%\setheadsepline{1.2pt}
%\usepackage{ifthen}
%\newboolean{SectionOnPage}
%\renewcommand{\sectionmark}[1]{\setboolean{SectionOnPage}{true}\markboth{#1}{#1}}
%\newcommand{\headerblackbox}{\colorbox{black}{\parbox[c][6mm]{0.1cm}{\ }}}
%\newcommand{\headerbluebox}[1]{\colorbox{HeadBlue}{\parbox[c][6mm]{1.8cm}{%
%%       \ifnum\value{section}>0
%%         \ifthenelse{true}{H}{#1\textbf{\begin{large}\color{white}{\sffamily{\arabic{section}}}\end{large}}}
%%       \else
%          #1 \textbf{\begin{large}\color{white}{\sffamily{\ }}\end{large}}
%%       \fi
%}}}
%\newcommand{\headerbluetext}[1]{\textbf{\begin{large}\color{HeadBlue}{#1}\end{large}}}
%%\newcommand{\headerNTB}{\headerbluetext{NTB}}
%\newcommand{\headerNTB}{\includegraphics[height=1.2cm]{ntb}}
%\newcommand{\headerleftmark}{\headerbluetext{\firstleftmark}}
%
%\lehead{\hspace*{5mm}\headerleftmark \AddToShipoutPicture*{%
%  \AtPageUpperLeft{\put(0,\LenToUnit{-13.4mm})}}}
%
%\rohead{\headerleftmark \hspace*{5mm} \AddToShipoutPicture*{%
%  \AtPageUpperLeft{\put(\LenToUnit{\paperwidth},\LenToUnit{-13.4mm})}}}
%
%
%%\rehead{\headerNTB}
%%\lohead{\headerNTB}
%\rehead{Seite \thepage{}}
%\setheadsepline{0.8pt}
%\renewcommand*{\chapterpagestyle}{scrheadings}
%%\setheadsepline{0.5pt}[\color{blue}]
%
%%Fusszeile
%%\ifoot{Marcel Gehrig, Egemen Yesil}
%%\cfoot{\today}
%\cfoot{- \thepage{} -}
%%\ofoot{\thepage{} / \ref{TotPages}}
%%\setfootsepline{0.5pt}

%%%%%%%%%%%%%%%%%%%%%%%%%
% Informationen         %
% über den Autor        %
%%%%%%%%%%%%%%%%%%%%%%%%%

\title{VT1: EEROS Sequencer 2017}
\author{Marcel Gehrig}
\date{Januar 2017}


%%%%%%%%%%%%%%%%%%%%%%%%%
% Pdf Einstellungen     %
%%%%%%%%%%%%%%%%%%%%%%%%%

% Paket für Links innerhalb des PDF Dokuments und Pdf Informationen
\definecolor{LinkColor}{rgb}{0,0,0}
\usepackage[
  pdftitle={VT1 EEROS Sequencer},
  pdfauthor={Marcel Gehrig},
  pdfcreator={Texmaker},
%  pdfsubject={VT1 EEROS Sequencer},
  pdfsubject=pdftitle,
  pdfkeywords={Schlussbericht,Vertiefungsarbeit,EEROS,Sequencer,C++,Roboter Steuerung}]{hyperref}
\hypersetup{colorlinks=true,
  linkcolor=LinkColor,
  citecolor=LinkColor,
  filecolor=LinkColor,
  menucolor=LinkColor,
  urlcolor=LinkColor}

% Packet um Dateien in das PDF einzubetten
\usepackage{attachfile}
\usepackage{pdfpages}
%%%%%%%%%%%%%%%%%%%%%%%%%
% Farben                %
%%%%%%%%%%%%%%%%%%%%%%%%%

\definecolor{Blue}{rgb}{0,0,1}
\definecolor{Red}{rgb}{1,0,0}
\definecolor{Green}{rgb}{0,1,0}

%%%%%%%%%%%%%%%%%%%%%%%%%%%
%%%%%%%%%%%%%%%%%%%%%%%%%%%
%% ACHTUNG:              %%
%% Informationen über    %%
%% den Autor müssen in   %%
%% folgenden Abschnitten %%
%% angepasst werden:     %%
%%                       %%
%% - PDF EINSTELLUNGEN   %%
%% - KOPF UND FUSSZEILE  %%
%% - AUTOR               %%
%%%%%%%%%%%%%%%%%%%%%%%%%%%

%Tiefe der nummerierten Ebene
\setcounter{tocdepth}{4} %Anzeigetiefe im Inhaltsverzeichniss
\setcounter{secnumdepth}{4} %Nummerierungstiefe
\renewcommand*{\chapterpagestyle}{scrheadings} %Kopfzeile auch bei \chapter
\renewcommand*{\chapterheadstartvskip}{\vspace*{-\topskip}}  %Abstand korrigieren von Kopfzeile zu \chapter
\setlength{\cftbeforetoctitleskip}{-2mm} 
\setlength{\cftaftertoctitleskip}{0.5mm}
\usepackage{placeins} %um floatbarrier zu benutzen
%\usepackage{mathtext}

%Pfad für die Bilder Festlegen
%\graphicspath{{Bilder/},{Icons/},{Logo/},{Template/},{Ismages/},{}}
\graphicspath{{images_P/},{}}