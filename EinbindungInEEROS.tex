\chapter{Einbindung in EEROS}


\section{CMAKE}
\subsection{Erkennen ob ROS installiert ist}
Diejenigen Teile von EEROS welche ROS-Bibliotheken verwenden, können nur kompiliert werden, wenn ROS auch auf dem System installiert ist.
CMAKE nutzt dafür den "\textit{find\_package()}-Befehl.
\textit{roscpp} ist nur ein  einzelnes \textit{Package} und nicht das ganze ROS-Framework.
Wenn aber \textit{roscpp} gefunden wird, kann davon ausgegangen werden, dass auch das restliche Framework installiert wurde.

%\lstset{language=cmake}
\lstset{language=c}
\begin{lstlisting}
message(STATUS "looking for package 'ROS'")
find_package(roscpp QUIET)
if (roscpp_FOUND)
	message( STATUS "-> ROS found")
	message( STATUS "roscpp_DIR:  " ${roscpp_DIR})
endif()
\end{lstlisting}

\subsection{Mögliche Probleme}
Bevor CMAKE das \textit{Package} finden kann, muss das \textit{Setup-Skript} von ROS ausgeführt werden.
Üblicher weise kann das Skript mit folgendem Befehl ausgeführt werden:\\
\textit{\$ source /opt/ros/kinetic/setup.bash}

