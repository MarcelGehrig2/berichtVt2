\chapter{Testing}
\section{Einleitung}
%TODO beschreibung
%TODO repos benamsen

\section{Tests für ''Essential Features''}
\subsection{Unabhängiger ROS-Knoten}
\subsubsection{Zu erfüllende Testbedingungen}
Eine C++-Applikation schreiben, die folgende Eigenschaften erfüllt:
\begin{enumerate}
\item Applikation meldet sich als ROS-Knoten an.
\item Applikation schickt eine \textit{ROS Log Statement}.
\end{enumerate}

\subsubsection{Testdurchführung}
\textbf{Repositories:} \\
\begin{tabular}
  { l						| l			 							l								 l								}
% Name						Repo   									Branch	     				Tag
  testAppVT2\_t1.0			& \textit{Repository}: eeros-framework	& \textit{Branch}: masster		& \textit{Tag}: Test001.0 		\\
\end{tabular}

\textbf{Ablauf: }
\begin{enumerate}
\item Den ROS Core mit \textit{\$ roscore} starten.
\item \textit{\$ rqt\_console} starten.
\item Mit dem Befehl \textit{\$ rosnode list} 
\item Testapplikation ''\textit{testAppVT2\_t1.0}'' starten.
\item Solange die Applikation läuft, muss bei \textit{\$ rosnode list} ein neuer Node aufgelistet sein. \\
\textbf{Ergebnis:} \checkmark
\item Bei der \textit{rqt\_console} ist mindestens eine neue \textit{Log-Message} von der Testapplikation erschienen. \\
\textbf{Ergebnis:} \checkmark
\item Nachdem die Testapplikation beendet wurde, ist der Knoten der Testapplikation unter \textit{\$ rosnode list} wieder verschwunden. \\
\textbf{Ergebnis:} \checkmark
\end{enumerate}


\subsection{CMAKE}
\subsubsection{Zu erfüllende Testbedingungen}
Eine Klasse in EEROS erstellen, die ROS verwendet und den EEROS Quellcode umschreiben, damit folgende Bedingungen erfüllt werden:
\begin{itemize}
\item Wenn ROS installiert ist, wird die neu geschriebene Klasse kompiliert und gegen die entsprechenden ROS-Bibliotheken gelinkt.
\item Wenn ROS nicht installiert ist, dann wird die neu geschriebene Klasse nicht kompiliert und die restlichen Teile von EEROS kompilieren fehlerfrei.
\end{itemize}

\subsubsection{Testdurchführung}
\textbf{Repositories:} \\
\begin{tabular}
  { l						| l			 							l								 l								}

% Name						Repo   									Branch Aufwand     				Tag
  EEROS\_t2.0				& \textit{Repository}: eeros-framework	& \textit{Branch}: ROSVt2		& \textit{Tag}: Test002.0 		\\
  EEROS-Applikation\_t2.0	& \textit{Repository}: testAppEEROSVt2	& \textit{Branch}: master		& \textit{Tag}: Test002.0 		\\
\end{tabular}

\textbf{Ablauf: } 
\begin{enumerate}
\item Den \textit{build} Ordner und den \textit{install} Ordner von ''\textit{EEROS\_t2.0} löschen.
\item \textit{CMAKE} ausführen, \textbf{ohne} dass vorher das \textit{Setup-Skript} von ROS ausgeführt wurde.
\item Wenn \textit{CMAKE} ausgeführt wird, erscheint unter anderem folgende Ausgabe: \\
\textit{-- looking for package 'ROS' \\
-- -> ROS NOT found} \\
\textbf{Ergebnis:} \checkmark
\item EEROS baut fehlerfrei und wird richtig installiert. \\
\textbf{Ergebnis:} \checkmark
\item Die EEROS-Testapplikation ''\textit{EEROS-Applikation\_t2.0}'' lässt sich \textbf{nicht} bauen, da ein \textit{Header file} von ROS fehlt. \\
\textbf{Ergebnis:} \checkmark
\item Den \textit{build} Ordner und den \textit{install} Ordner von ''\textit{EEROS\_t2.0} löschen.
\item \textit{CMAKE} ausführen, \textbf{nachdem} das \textit{Setup-Skript} von ROS ausgeführt wurde.
\item Wenn \textit{CMAKE} ausgeführt wird, erscheint unter anderem folgende Ausgabe: \\
\textit{-- looking for package 'ROS' \\
-- -> ROS found} \\
\textbf{Ergebnis:} \checkmark
\item EEROS baut fehlerfrei und wird richtig installiert. \\
\textbf{Ergebnis:} \checkmark
\item Die EEROS-Testapplikation ''\textit{EEROS-Applikation\_t2.0}'' lässt sich bauen. \\
\textbf{Ergebnis:} \checkmark
\item Den ROS Core mit \textit{\$ roscore} starten.
\item \textit{\$ rqt\_console} starten.
\item Die EEROS-Testapplikation lässt sich mit \textit{\$ sudo -E ./testappEEROSVT2 } starten. \\
\textbf{Ergebnis:} \checkmark
\item Bei der \textit{rqt\_console} ist mindestens eine neue \textit{Log-Message} von der Testapplikation erschienen. \\
\textbf{Ergebnis:} \checkmark
\end{enumerate}



\section{Fernsteuerung}
\subsection{Einfacher ROS-Knoten}
\subsubsection{Zu erfüllende Testbedingungen}
In EEROS einen Block für das \textit{Control System} erstellen, welcher das \textit{Topic} vom \textit{turtle\_teleop\_key} einlesen kann.
\begin{itemize}
\item Eine EEROS-Testapplikation verwendet einen dafür vorgesehenen Block von EEROS, um die vom \textit{turtle\_teleop\_key} publizierten \textit{Messages} anzuzeigen.
\end{itemize}

\subsubsection{Testdurchführung}
\textbf{Repositories:} \\
\begin{tabular}
  { l						| l			 							l								 l								}

% Name						Repo   									Branch Aufwand     				Tag
  EEROS\_t3.0				& \textit{Repository}: eeros-framework	& \textit{Branch}: ROSVt2		& \textit{Tag}: Test003.0 		\\
  EEROS-Applikation\_t3.0	& \textit{Repository}: testAppEEROSVt2	& \textit{Branch}: master		& \textit{Tag}: Test003.0 		\\
\end{tabular}

\textbf{Ablauf: } 
\begin{enumerate}
\item Den ROS Core mit \textit{\$ roscore} starten.
\item Testapplikation ''\textit{EEROS-Applikation\_t3.0}'' in einem neuen Terminal starten.
\item Den \textit{Turtlesim} Knoten mit \textit{\$ rosrun turtlesim turtle\_teleop\_key} starten.
\item Für die vier Pfeiltaste muss beim Terminal von der Testapplikation eine entsprechende Ausgabe erscheinen. \\
\textbf{Ergebnis:} \checkmark
\item Beide Applikationen beenden.
\item Den \textit{Turtlesim} Knoten mit \textit{\$ rosrun turtlesim turtle\_teleop\_key} starten.
\item Testapplikation ''\textit{EEROS-Applikation\_t3.0}'' in einem neuen Terminal starten.
\item Das Teminal mit dem \textit{Turtlesim} Knoten anwählen.
\item Für die vier Pfeiltaste muss beim Terminal von der Testapplikation eine entsprechende Ausgabe erscheinen. \\
\textbf{Ergebnis:} \checkmark
\item Den \textit{Turtlesim} Knoten beenden.
\item Den \textit{Turtlesim} Knoten mit \textit{\$ rosrun turtlesim turtle\_teleop\_key} neu starten.
\item Für die vier Pfeiltaste muss beim Terminal von der Testapplikation eine entsprechende Ausgabe erscheinen. \\
\textbf{Ergebnis:} \checkmark
\end{enumerate}




%\textbf{Repositories:} \\
%\begin{tabular}
%  { l						| l			 							l								 l								}
%
%% Name						Repo   									Branch Aufwand     				Tag
%  EEROS\_t2.0				& \textit{Repository}: eeros-framework	& \textit{Branch}: ROSVt2		& \textit{Tag}: Test002.0 		\\
%  EEROS-Applikation\_t2.0	& \textit{Repository}: testAppVt2		& \textit{Branch}: master		& \textit{Tag}: Test002.0 		\\
%\end{tabular}
%
%\textbf{Ablauf: } \\
%\begin{enumerate}
%\item 
%%\textbf{Ergebnis:} \checkmark
%\end{enumerate}
