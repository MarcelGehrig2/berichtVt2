\chapter{Einleitung}
%TODO arbeitsablauf

\section{Repositories}
Für diese Arbeit werden öffentliche Git-Repositories verwendet.
Alle Repositories werden auf GitHub gehostet.

Das Haupt-Repository enthält alle digitalen Daten dieser Arbeit inklusive aller anderen Sub-Repositories (Submodule).
Mit folgendem Befehl können alle Repositories auf einmal geklont werden:

\textit{git clone --recursive -j8 https://github.com/MarcelGehrig2/VT2.git}

\begin{tabular}
  { l						l			 												l						}

% Name						URL   														Branch     			
  \textbf{Name}				& \textbf{URL}												& \textbf{Branch}		\\
  VT2						& https://github.com/MarcelGehrig2/VT2.git					& master				\\
  EEROS						& https://github.com/MarcelGehrig/eeros-framework			& master				\\
  EEROSTestApp				& https://github.com/MarcelGehrig2/testAppEEROSEVT2.git		& master		 		\\
  SimpleROSNode				& https://github.com/MarcelGehrig2/simpleROSNodeVt2.git		& master				\\
  Bericht					& https://github.com/MarcelGehrig2/berichtVt2.git			& master				\\
\end{tabular}
%TODO cmake


\section{Aufgabenstellung}
%TODO

\section{Arbeitsablauf}
Der Arbeitsablauf der in dieser Arbeit eingehalten wurde, ist im Anhang \ref{anhangAblaufSoftwareentwicklung} als Flow Chart dargestellt.

Erst wurden sogenannte \textit{Features} identifiziert.
\textit{Features} erweitern die Software um eine bestimmte Funktion.
Im Kapitel \ref{chapter:featureList} werden alle \textit{Features} aufgelistet und nach Signifikanz und Arbeitsaufwand bewertet.

Für jeweils ein \textit{Feature} wurde im Kapitel \ref{chapter:testing} ein Test geschrieben.
Die Software wurde dann, mit Hilfe des im Anhang \ref{anhangAblaufCodesegement} beschriebenen Arbeitsablauf, um das entsprechende \textit{Feature} erweitert.
Wenn der oben beschriebene Test erfolgreich durchlief, wurde, falls sinnvoll, das Feature im englischen Wiki im Kapitel \ref{chapter:wiki} dokumentiert.

In einem Echtzeitsystem wie EEROS ist es essentiell, das der Ablauf nicht blockiert wird.
Deshalb wurde der Zeitbedarf von einigen Funktionsaufrufen gemessen und im Kapitel \ref{chapter:performanceTests} dokumentiert.

Natürlich treten bei so einer Arbeit diverse Probleme auf.
Einige dieser Probleme, besonders solche die möglicherweise in Zukunft wieder auftreten können, wurden im Kapitel \ref{chapter:problembehebung} dokumentiert.


