\chapter{Einleitung}
\label{chapter:einleitung}
\section{Repositories}
Für diese Arbeit werden öffentliche Git-Repositories verwendet.
Alle Repositories werden auf GitHub gehostet.

Das Haupt-Repository enthält alle digitalen Daten dieser Arbeit inklusive aller anderen Sub-Repositories (Submodule) und gilt in dieser Arbeit als digitaler Anhang.
Der aktuellste Commit zum Zeitpunkt der Abgabe wird mit dem Tag \textit{FinalRelease} versehen.
Mit folgendem Befehl können alle Repositories auf einmal geklont werden:

\textit{git clone --recursive -j8 https://github.com/MarcelGehrig2/VT2.git}

\begin{tabular}
  { l						l			 												l						}

% Name						URL   														Branch     			
  \textbf{Name}				& \textbf{URL}												& \textbf{Branch}		\\
  VT2						& https://github.com/MarcelGehrig2/VT2.git					& master				\\
  EEROS						& https://github.com/MarcelGehrig/eeros-framework			& master				\\
  EEROSTestApp				& https://github.com/MarcelGehrig2/testAppEEROSEVT2.git		& master		 		\\
  SimpleROSNode				& https://github.com/MarcelGehrig2/simpleROSNodeVt2.git		& master				\\
  Bericht					& https://github.com/MarcelGehrig2/berichtVt2.git			& master				\\
  Bericht von Manuel Ilg	& https://github.com/manuelilg/vt2\_bericht					& master				\\
\end{tabular}
%TODO cmake


\section{Zusammenarbeit}
Manuel Ilg hat in seiner Vertiefungsarbeit \textit{\char"22ROS als unterstützendes Werkzeug für EEROS-Applikationen\char"22} aus dem Jahre 2017 eine Regelstrecke in Gazebo simuliert.	%TODO titel
Er hat nicht nur ein Modell der Regelstrecke in Gazebo erstellt sondern auch bei der Entwicklung der Kommunikation und Synchronisation mit EEROS aktiv mitgeholfen.


\section{Aufgabenstellung}
Das Roboterframework EEROS\footnote{http://eeros.org/wordpress/} ist eine open source Software, die an der NTB entwickelt wurde und immer noch weiterentwickelt wird.
Es handelt sich dabei um ein Framework, mit dem sich Steuerungen für Roboter realisieren lassen.
Das echtzeitfähige Kontrollsystem erlaubt Regelungen mit komplexen Kinematiken zu verwirklichen.

ROS\footnote{http://www.ros.org/} ist eine Sammlung von Softwarebibliotheken und Werkzeugen, die die Entwicklung von Roboterprojekten unterstützen.
ROS hat eine grosse Community, die bereits viele nützliche Tools veröffentlicht hat.
In ROS wird über sogenannte \textit{Topics} kommuniziert, die einen standardisierten Weg bieten, damit die verschiedenen Tools und Sensoren miteinander kommunizieren können.
Es ist aber nicht möglich, eine echtzeitfähige Regelung zu implementieren.

Ziel dieser Arbeit ist es, für EEROS eine Anbindung an das ROS Netzwerk zu implementieren.
Es soll möglich sein, Daten von einem ROS \textit{Topic} zu lesen und im Kontrollsystem zu verwenden.
Ebenfalls sollen Daten aus dem Kontrollsystem auf einem \textit{Topic} veröffentlicht werden können, damit sie mit bestehenden ROS Tools visualisiert und ausgewertet werden können.

Um eine EEROS Applikation ohne Hardware testen zu können, soll die Hardware mit Gazebo simuliert werden können.
Mit solchen Simulationen sollen nicht nur Abläufe getestet werden können, sondern auch zeitkritische Regelkreise.


\section{Arbeitsablauf}
Der Arbeitsablauf, der in dieser Arbeit eingehalten wurde, ist im Anhang \ref{anhangAblaufSoftwareentwicklung} als Flow Chart dargestellt.

Erst wurden sogenannte \textit{Features} identifiziert.
\textit{Features} erweitern die Software um eine bestimmte Funktion.
Im Kapitel \ref{chapter:featureList} werden alle \textit{Features} aufgelistet und nach Signifikanz und Arbeitsaufwand bewertet.

Für jeweils ein \textit{Feature} wurde im Kapitel \ref{chapter:testing} ein Test geschrieben.
Die Software wurde dann, mit Hilfe des im Anhang \ref{anhangAblaufCodesegement} beschriebenen Arbeitsablaufs, um das entsprechende \textit{Feature} erweitert.
Wenn der oben beschriebene Test erfolgreich durchlief, wurde, falls sinnvoll, das Feature im englischen Wiki im Kapitel \ref{chapter:wiki} dokumentiert.

In einem Echtzeitsystem wie EEROS ist es essentiell, dass der Ablauf nicht blockiert wird.
Deshalb wurde der Zeitbedarf von einigen Funktionsaufrufen gemessen und im Kapitel \ref{chapter:performanceTests} dokumentiert.

Natürlich treten bei einer solchen Arbeit diverse Probleme auf.
Einige dieser Probleme, besonders solche die möglicherweise in Zukunft wieder auftreten können, wurden im Kapitel \ref{chapter:problembehebung} dokumentiert.

Im Kapitel \ref{chapter:ros} sind kurz die wichtigsten Informationen über ROS zusammengefasst.
