\chapter{ROS}
\label{chapter:ros}

\section{Hilfreiche ROS-Werkzeuge}

\begin{itemize}
\item \textbf{rqt:} Eine Sammlung von diversen ROS Tools.
\item \textbf{rqt-multiplot:} Sehr gutes Programm um mehrere Signale in einem oder mehreren Graphen darzustellen. Die gemessene Daten lassen sich als Text Datei speichern.
\item \textbf{Gazebo:} Software im Regelstrecken zu simulieren. Mehr dazu in der Arbeit von Manuel Ilg\footnote{\char"22ROS als unterstützendes Werkzeug für EEROS-Applikationen\char"22 aus dem Jahr 2017}.
\item \textbf{rviz:} Grafische Oberfläche um Zustände von einem Roboter darzustellen.
\end{itemize}


\section{Allgemeine Funktionsweise von ROS}

\begin{itemize}
\item Wenn \textit{Messages} von einem \textit{Node} schneller abgefragt werden, als neue \textit{Messages} veröffentlicht werden, dann wird die zuletzt veröffentlichte \textit{Message} mehrmals zurückgegeben.
\item Wenn \textit{Messages} schneller \textit{published} als abgeholt werden, dann werden die \textit{Messages} in einem \textit{Buffer}, der sogenannten '\textit{Message Queue}', zwischengespeichert. Die Grösse der \textit{Message Queue} wird definiert, wenn sich ein \textit{Node} (\textit{Publisher} oder \textit{Subscirber}) bei einem \textit{Topic} anmeldet. Der \textit{Publisher} und der \textit{Subscriber} haben je einen unabhängige \textit{Buffer}. Der \textit{Subscriber} erhält immer die älteste, nicht abgeholte \textit{Message} zuerst.
\item Ist der \textit{Buffer} vom \textit{Publisher} voll, dann werden die ältesten \textit{Messages} überschrieben.
\end{itemize}


\section{Implementation in ROScpp}
\begin{itemize}
\item '\textit{ros::spinOnce()}' führt für jede \textit{Message} in der \textit{Message Queue} die \textit{Callback Function} einmal aus. Die älteste \textit{Message} wird zuerst verarbeitet. Sobald die letzte \textit{Message} verarbeitet wurde, endet der Befehl.
\item '\textit{ros::spin()}' funktioniert wie \textit{ros::spinOnce()}, aber der Befehl blockiert weiter, wenn die letzte \textit{Message} verarbeitet wurde. Wird eine neue \textit{Message} auf dem \textit{Topic} veröffentlicht, wird sofort die \textit{Callback Function} ausgeführt, da \textit{ros::spin()} immer auf neue \textit{Messages} wartet.
\item '\textit{ros::getGlobalCallbackQueue()->callAvailable()}' ist von der Funktion her identisch wie \textit{ros::spinOnce()}. Nur der Name ist anders.
\item '\textit{ros::getGlobalCallbackQueue()->callOne()} führt die \textit{Callback Function} nur für die älteste \textit{Message} aus.
\end{itemize}


\section{Debugging Hilfen}
\begin{itemize}
\item Wenn der Logger '\textit{ros.roscpp}' eines \textit{Subsribers} auf den Level '\textit{Debug}' gesetzt wird, dann werden Warnungen im Stil von ''\textit{Incomming queue was full for topic ...}'' ausgegeben, wenn der \textit{Subsciber} die \textit{Messages} nicht schnell genug verarbeiten kann und der Buffer überfüllt ist.
\item Eine ähnliche Warnung wird beim \textit{Logger} '\textit{ros.roscpp}' eines \textit{Publishers} ausgegeben, wenn die \textit{Messages} nicht schnell genug geschickt werden können. Dies wäre zum Beispiel der Fall, wenn die Netzwerkverbindung zu langsam ist.
\end{itemize}