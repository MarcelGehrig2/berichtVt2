\chapter{Problembehebung}
\label{chapter:problembehebung}

\section{ROS wird von CMAKE nicht gefunden}
\subsection{Problembeschreibung}
Beim kompilieren von EEROS wird ROS nicht gefunden.
Wenn CMAKE ausgeführt wird, werden die Packages von ROS nicht gefunden.

\subsection{Mögliche Ursachen}
\begin{enumerate}
\item ROS wurde auf der Maschine nicht installiert.
\item Der \textit{setup.bash} Skript von ROS wurde nicht ausgeführt, bevor CMAKE aufgerufen wird.
In diesem Fall stehen CMAKE die benötigten Umgebungsvariablen nicht zur Verfügung.
Dies ist auch der Fall, wenn CMAKE in \textit{Qt Creator} ausgeführt wird.
\end{enumerate}

\subsection{Lösung}
\begin{enumerate}
\item ROS installieren.
\item Sicherstellen, dass der \textit{setup.bash} Skript von ROS ausgeführt wird, bevor CMAKE aufgerufen wird.
Typischerweise wird dieser Skript aus dem \textit{~/.bashrc} Skript automatisch ausgeführt, wenn eine Konsole geöffnet wird.
\item Wird CMAKE aus einer Entwicklungsumgebung (z.B. \textit{Qt Creator}) ausgeführt, dann muss die Entwicklungsumgebung aus dem Terminal und nicht per Icon gestartet werden.
Wird die Software per Icon gestartet, dann wird vorher der \textit{~/.bashrc} Skript nicht ausgeführt und die benötigten ROS-Umgebungsvariablen stehen CMAKE nicht zur Verfügung.
Wird \textit{QT Creator} aus einem Terminal gestartet, dann wird der \textit{~/.bashrc} Skript wie gewünscht ausgeführt.
Mit dem folgenden Befehl kann der \textit{QT Creator} aus dem Terminal gestartet werden (der genaue Pfad hängt von der Version ab):\\
\textit{\$ ~/Qt5.7.0/Tools/QtCreator/bin/qtcreator \&}
\end{enumerate}


\section{Probleme mit ROS, wenn sudo verwendet wird}
\subsection{Problembeschreibung}
Wenn eine Applikation (EEROS-Applikation oder unabhängige Applikation) gestartet wird, welche ROS verwendet, erscheint folgende Fehlermeldung:


\texttt{
[FATAL] [1494864699.611423845]: ROS\_MASTER\_URI is not defined in the environment. Either type the following or (preferrably) add this to your ~/.bashrc file in order set up your local machine as a ROS master:
}

\texttt{
export ROS\_MASTER\_URI=http://localhost:11311
}

\texttt{
then, type 'roscore' in another shell to actually launch the master program.
}

\subsection{Mögliche Ursache}
Der \textit{sudo}-Befehl übergibt die Umgebungsvariablen des Prozess nicht, aus dem er gestartet wird.
Deshalb stehen die Umgebungsvariablen, welche vom \textit{setup.sh}-Skript definiert werden, dem ROS-Programm nicht zur Verfügung.

\subsection{Lösung}
Verwende: \\
\textit{\$ sudo -E ./applikation} \\
anstelle von: \\
\textit{\$ sudo ./applikation}.



\section{Problem beim Speichern von Daten von einer ROS-Message in ein EEROS-Signal}
%TODO

\subsection{Problembeschreibung}
%TODO wann tritt das auf
Fehlermeldung:

\texttt{
.../Matrix.hpp:46: error: no matching function for call to \\
'forward(const std::vector<float>\&)' \\
\-\hspace{2cm} Matrix(const S... v) : value{std::forward<const T>(v)...} 
}

\subsection{Mögliche Ursachen}
Der \textit{Vector} von der \textit{ROS-Message} ist nicht vom gleichen Typ wie der Vektor vom Ausgangssignal.

\subsection{Lösung}
Der \textit{Vecctor} kann folgendermassen zum gewünschten Typ umgewandelt werden:
\lstset{language=C++}
\begin{lstlisting}
std::vector<double> tmp( msg.axes.begin(), msg.axes.end() );	// cast because axes is a float32 vector
axesValue.setCol(0, tmp);	// double vector
axesOutput.getSignal().setValue(axesValue); 
\end{lstlisting}