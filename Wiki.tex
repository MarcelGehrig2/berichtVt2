\chapter{Wiki}
\section{Einleitung}
Das folgende Kapitel ist so geschrieben, dass es möglichst einfach in das EEROS-Wiki\footnote{http://wiki.eeros.org/} übernommen werden kann.
Da das EEROS-Wiki in englischer Sprache geschrieben ist, sind auch die folgenden Kapitel in Englisch verfasst.


\section{Using ROS with EEROS}	%TODO referenz auf repos? ros-eeros
There are three different use cases for using ROS with EEROS.
The easiest way is to use ROS only as a debugging tool like it is described in chapter \ref{rosAsDebuggingTool}.

If you want to subscribe to a node or publish to a ROS node, the implementation is a bit more sophisticated.
You can either add a ROS block to the control system, like it is described in chapter \ref{rosBlocks}, or you can use the EEROS HAL.
With the HAL you can easily switch between publishing an output value to a ROS node and writing an output value directly to hardware.
To use the HAL you may have to adapt the wrapper library \textit{ros-eeros}.
Chapter \ref{rosHal} describes in detail, how you can adapt the wrapper library.

Independent of which way you plan to use ROS-EEROS you first have to do some preparations.
Chapter \ref{preparations} describes all needed steps to prepare your system.

You can find an example application in the eeros source code.
The example is located under \textit{eeros-framework/examples/ros}


\section{Preparations}
\label{preparations}
\subsection{Configure toolchain}
\label{configureToolchain}
To build an EEROS application with ROS, ROS ''kinetic'' needs to be installed\footnote{http://wiki.ros.org/kinetic/Installation} on the developer machine and on the target machine.
Before a ROS application can be started, you need to run the \textit{setup.bash} script of ROS.
The same applies for building the EEROS library with ROS support and for building an EEROS application with ROS support.

In EEROS \textit{CMAKE} is used to build an application.
If the EEROS application has dependencies on ROS, the \textit{setup.bash} script of ROS has to be executed before \textit{CMAKE} is called.
If an IDE like ''\textit{Qt Creator}'' is used, the application has to be started from a terminal.
\textit{CMAKE} will not find the ROS libraries if \textit{QT Creator} is launched from a desktop icon.

\subsection{Configure the CMAKE file}
The following example shows a \textit{CMAKE} file for a simple EEROS application with ROS.

Don't forget to add the ROS libraries to your application:
\begin{snugshade*}
	target\_link\_libraries(helloWorld eeros \${ROS\_LIBRARIES})
\end{snugshade*}

%\lstset{language=cmake}
\lstset{language=c}
\begin{lstlisting}
cmake_minimum_required(VERSION 2.8)
 
project(helloWorld)


## ROS
## ////////////////////////////////////////////////////////////////////////
message(STATUS "looking for package 'ROS'")
find_package( roslib REQUIRED )
if (roslib_FOUND)
	message( STATUS "-> ROS found")
	add_definitions(-DROS_FOUND)
	include_directories( "${roslib_INCLUDE_DIRS}" )
	message( STATUS "roslib_INCLUDE_DIRS: " ${roslib_INCLUDE_DIRS} )
	list(APPEND ROS_LIBRARIES "${roslib_LIBRARIES}")
	find_package( rosconsole REQUIRED)
	list(APPEND ROS_LIBRARIES "${rosconsole_LIBRARIES}")
	find_package( roscpp REQUIRED )
	list(APPEND ROS_LIBRARIES "${roscpp_LIBRARIES}")
else()
	message( STATUS "-> ROS NOT found")
endif()


## EEROS
## ////////////////////////////////////////////////////////////////////////
find_package(EEROS REQUIRED)
include_directories(${EEROS_INCLUDE_DIR})
link_directories(${EEROS_LIB_DIR})


## Application
## ////////////////////////////////////////////////////////////////////////
set(CMAKE_CXX_FLAGS "${CMAKE_CXX_FLAGS} -std=c++11")
 
add_executable(helloWorld
	main.cpp	
)


target_link_libraries(helloWorld eeros ${ROS_LIBRARIES})

\end{lstlisting}

\subsection{Initialize a ROS node in your EEROS application}
Add the ROS header file to the main source file of your application:
\lstset{language=c}
\begin{lstlisting}
	#include <ros/ros.h>
\end{lstlisting}

Initialize the ROS node in your main function and pass a reference of the ROS node handler to your control system:	
\lstset{language=c}
\begin{lstlisting}
...
int main(int argc, char **argv) {
	...
	// ROS
	// ////////////////////////////////////////////////////////////////////////
	char* dummy_args[] = {NULL};
	int dummy_argc = sizeof(dummy_args)/sizeof(dummy_args[0]) - 1;
 	ros::init(dummy_argc, dummy_args, "EEROSNode");
	ros::NodeHandle rosNodeHandler;
	log.trace() << "ROS node initialized.";
	
	// Control System
	// ////////////////////////////////////////////////////////////////////////
	MyControlSystem controlSystem(dt, rosNodeHandler);
	...
\end{lstlisting}

If you want to register a signal handler, you have to register it after the ROS node is initialized.
Don't worry.
The ROS node gets properly shut down as soon as the node handler is destroyed at the end of the application.


\subsection{ROS time}
In EEROS the timestamp for output signals is created with the clock \textit{CLOCK\_MONOTONIC\_RAW} of the system.
To get the time you can use:
\lstset{language=c}
\begin{lstlisting}
eeros::System::getTimeNs();
\end{lstlisting}

If you want, you also can use the ROS clock. 
This can be usefull, if you are simulating with Gazebo.
To switch to ROS time, call
\lstset{language=c}
\begin{lstlisting}
eeros::System::useRosTime();
\end{lstlisting}
before you run the executor.


\subsection{Starting an EEROS application which uses ROS}
An EEROS application needs to be started with super user privileges.
ROS needs some system variables, like \textit{ROS\_MASTER\_URI}, which are defined by the \textit{setup.bash} script of ROS.
To forward these variables to the super user process, the option \textit{-E} has to be used. Here an example:
\begin{snugshade*}
	\$ sudo -E ./applikation
\end{snugshade*}


\section{Use ROS as debugging tool}
\label{rosAsDebuggingTool}
\subsection{Introduction}
The \textit{RosBlockPublisher.hpp} is an easy way to monitor signals in the EEROS control system.
It can publish the value and the timestamp of a simple signal every iteration of the control system.
Period times of \unit[1]{msec} are no problems.
Althought it publishes every iteration with the right order, the published messages arrive not in real-time!

You can view and store the published messages with ROS tools like \textit{multiplot}\footnote{http://wiki.ros.org/rqt\_multiplot}.

Install \textit{multiplot}:
\begin{snugshade*}
	sudo apt-get install ros-kinetic-rqt
\end{snugshade*}

To run \textit{multiplot} open \textit{rqt} and add the plugin \textit{Plugins/Visualization/Multiplot}.

\subsection{Implementation}
Add the header file of your control system:
\lstset{language=c}
\begin{lstlisting}
	#include <eeros/control/ROS/RosBlockPublisherDouble.hpp>
\end{lstlisting}

You can now create a \textit{RosBlockPublisherDouble} EEROS block.
This block accepts only EEROS signals with the type ''\textit{double}''.
\lstset{language=c}
\begin{lstlisting}
	RosBlockPublisherDouble		debugOut0;
\end{lstlisting}

Call the constructor of the block with the topic name, which it should publish to.
\lstset{language=c}
\begin{lstlisting}
class MyControlSystem {
public:
	MyControlSystem(double ts, ros::NodeHandle& rosNodeHandler):
		dt(ts),
		rosNodeHandler(rosNodeHandler),
		debugOut0(rosNodeHandler, "debugNode/debugOut0"),
		...
}
\end{lstlisting}

To publish a certain signal simply connect it to the block and run the block.
\lstset{language=c}
\begin{lstlisting}
class MyControlSystem {
	...
	debugOut0.getIn().connect(analogIn0.getOut());
	...
	timedomain.addBlock(debugOut0);
	...
\end{lstlisting}

\subsection{Limitations}
The \textit{RosBlockPublisherDouble} block can only publish EEROS signals with a single \textit{double} value.
The block publishes a ROS message of type \textit{sensor\_msgs/FluidPressure.h}.
This rather strange message type was chosen, because it contains a member of type \textit{double} and a header with timestamp and not much unused overhead.

A custom message type with only a header, a member of type \textit{doble} and an appropriate name could be createt with catkin.
But this could lead to compatibility issues with third party applications like \textit{rqt}.


\section{EEROS with ROS blocks as interface}
\label{rosBlocks}
\subsection{Introduction}
This method is rather similar to the use of the \textit{RosBlockPublisherDouble} described in the last chapter.
This chapter describes, how you can create your own ROS block in EEROS.
If you build your own ROS block, you can define the type of the EEROS signal and the type of the ROS message yourself.

A publisher block takes one or multiple EEROS signals and publishes the data to a ROS node.
A subscribe subscribes to a ROS node and creates an EEROS signal.

\subsection{Implementation}
This section describes how you can use the \textit{RosBlockPublisher\_SensorMsgs\_LaserScan} block, which is included with the \textit{eeros-framework}.
You can build your own ROS block like it is described in section \ref{createRosBlock}.

Add the header file to your control system:
\lstset{language=c}
\begin{lstlisting}
	#include <eeros/control/ROS/RosBlockSubscriber_SensorMsgs_LaserScan.hpp>
\end{lstlisting}

You can now create a \textit{RosBlockSubscriber\_SensorMsgs\_LaserScan} EEROS block.
This block has 9 different input signals.
One EEROS signal for every member of the ROS message with type \textit{sensor\_msgs::LaserScan}\footnote{http://docs.ros.org/jade/api/sensor\_msgs/html/msg/LaserScan.html}. 
Most members of the ROS message are from type \textit{double}.
The members \textit{ranges} and \textit{intensities} have variable length.
EEROS matrices on the other hand, have to have a fixed length.
This is why the types of \textit{ranges} and \textit{intensities} have to be declared, before a \textit{RosBlockSubscriber\_SensorMsgs\_LaserScan} can be declared.
This particular example creates EEROS matrices with the dimensions 5x1.
\lstset{language=c}
\begin{lstlisting}
	typedef eeros::math::Matrix< 5, 1, double >		TRangesOutput;
	typedef eeros::math::Matrix< 5, 1, double >		TIntensitiesOutput;
	RosBlockSubscriber_SensorMsgs_LaserScan<TRangesOutput, TIntensitiesOutput>	laserScanIn;
\end{lstlisting}

Call the constructor of the block with the ROS node handler, the topic name and the queue size and if only the newest messages should be called.
The queue size is the size of the ROS buffer.
If you choose to all the newest messages, only the newest message of each subscriber will be converted to an EEROS message and older messages can be discarded.
\lstset{language=c}
\begin{lstlisting}
class MyControlSystem {
public:
	MyControlSystem(double ts, ros::NodeHandle& rosNodeHandler):
		dt(ts),
		rosNodeHandler(rosNodeHandler),
		...
		laserScanIn (rosNodeHandler, "/rosNodeTalker/TestTopic3", 100, false),
}
\end{lstlisting}

The block can now be used like every other EEROS block.
If it is a publisher it has EEROS signal input and publishes to a ROS topic.
If it is a subscriber it has EEROS signal outputs and reads from a ROS topic.


\subsection{Creating a new ROS subscriber block}
\label{createRosBlock}
You can use the \textit{RosBlockSubscriber\_SensorMsgs\_LaserScan} block as an example.
The block is included in the \textit{eeros-framework}.

\textbf{Creating a new Block}
\label{creatingANewBlock}
\begin{enumerate}[\hspace{0.5cm}{A}-1]
\item Include the header file of the ROS message.
\item Define the type of the ROS message.
\item Name your block and create the constructor. Copy the type definition.
\item Add a new EEROS output for every data field you want to read -> see below.
\end{enumerate}

\textbf{Adding a new data field to an EEROS output}
\begin{enumerate}[\hspace{0.5cm}{B}-1]
\item Create EEROS outputs.
\item Add a 'getOutput()' function to each output.
\item Set the timestamp of all EEROS signal. You can use the system time or the timestamp of the ROS message.
\item For data fields of variable length use EEROS matrices -> see below.
\end{enumerate}

\textbf{Adding a new data field of variable length and an EEROS matrix output}
\begin{enumerate}[\hspace{0.5cm}{C}-1]
\item Create the template definition. Each EEROS matrix output needs its own type.
\item Create a 'value' and an 'output' variable for each EEROS matrix output.
\item Add a 'getOutput()' function for each EEROS matrix output.
\item Convert the vector of the ROS message to a <double>-vector.
\item Fill the vector into an EEROS matrix.
\item Fill the EEROS matrix in an EEROS signal.
\end{enumerate}


\subsection{Creating a new ROS publisher block}
You can use the \textit{RosBlockPublisher\_SensorMsgs\_LaserScan} block as an example.
The block is included in the \textit{eeros-framework}.
%TODO einleitung?

\textbf{Creating a new Block}\\
Identical procedure to \ref{creatingANewBlock} ''Creating a new Block'';

\textbf{Adding a new data field to an EEROS input}
\begin{enumerate}[\hspace{0.5cm}{B}-1]
\item Create EEROS inputs.
\item Add a 'getInput()' function for each input.
\item If available, set time in msg header.
\item Check if EEROS input is connected. Cast the data. Assign casted data to ROS message field.
\item For data fields of variable length use EEROS matrices -> see below.
\end{enumerate}

\textbf{Adding a new data field of variable length and an EEROS matrix input}
\begin{enumerate}[\hspace{0.5cm}{C}-1]
\item Create the template definition. Each EEROS matrix input needs its own type.
\item Create a 'value' and an 'output' variable for each EEROS matrix output.
\item Add a 'getInput()' function for each EEROS matrix Input.
\item Check if EEROS input is connected.
\item Get the vector from the EEROS input.
\item Cast the vector and assign it to the appropriate ROS data field.
\end{enumerate}



\section{EEROS HAL with ROS}
\label{rosHal}
\subsection{Introduction}
The wrapper library ''\textit{ros-eeros}'' is used to connect the EEROS HAL with ROS topics.
The EEROS HAL digital and analogue inputs and outputs can be defined with an *.json file.
If you want to test your application with a \textit{Gazebo} simulation, you can define your inputs and outputs as ROS topics to connect your application with the simulation completely without any real hardware.
To use your application with hardware, you can, for example, use the wrapper library \textit{comedi-eeros} or \textit{flink-eeros}.
If you adapt the *.json file correctly, your application should now run on hardware with real encoders and motors without any problems.

It is also possible to use ROS-topics alongside real hardware.
You can use \textit{comedi-eeros} to read an encoder and set a control value for a motor.
At the same time, you can publish the same values to ROS topics to monitor them with \textit{Gazebo} (if you have a model of your robot) or you can monitor the values with a ROS tool like ''\textit{rqt}'' and display them in a graph with \textit{Multiplot} or as numbers with \textit{TopicMonitor}.
With \textit{Multiplot} you can also store the values in a text file.

There are hundreds of different message types in ROS and it is possible to create custom types.
Because every message type has to be handled differently, only a few are supported by default.
But the wrapper library can easily be extended to support additional message types.
Chapter \ref{sectionImplementMsgType} describes how you can add a new ROS msg type to the library.


\subsection{The *.json file}
Table \ref{tableKeyValueEeros} shows the most important key-value pairs for using the HAL with ROS.

\begin{table}[]
\centering
\caption{Most important key-value pairs for ros-eeros}
\label{tableKeyValueEeros}
\begin{tabular}{@{}lll@{}}
\toprule
Key                 & Typical value                         & Description              \\ \midrule
library             & libroseeros.so                        & Wrapper library  for ROS \\
devHandle           & testNodeHAL                           & ROS node created by HAL  \\
type                & AnalogIn / AnalogOut / DigIn / DigOut & Type of input / output   \\ 
additionalArguments & 'see next table'                      & 'see next table'         \\ \bottomrule
\end{tabular}
\end{table}

The \textit{additionalArguments} are special arguments which are parsed in the wrapper library \textit{ros-eeros}.
These arguments contain additional information which are necessary to communicate with a ROS network.
All arguments are separated with a semicolon.
The available arguments are listed in table \ref{tableAdditionalArgumentsEeros}.

An example for an additional argument could be:

\begin{snugshade*}
\textit{\char"22additionalArguments\char"22: \char"22topic=/testNode/TestTopic3; msgType=sensor\_msgs::LaserScan; \\
dataField=scan\_time; callOne=false; queueSize=100\char"22,}
\end{snugshade*}


Table \ref{tableAdditionalArgumentsEeros} shows all currently available \textit{additionalArguments}.
\textbf{Topic} and \textbf{msgType} are mandatory arguments.

\begin{table}
\centering
\caption{Additional arguments specific for ros-eeros}
\label{tableAdditionalArgumentsEeros}
\begin{tabular}{@{}lll@{}}
\toprule
Key        & Typical value           & Description                                \\ \midrule
\textbf{topic}      & /testNode/TestTopic1    & Topic to listen / subscribe                \\
\textbf{msgType }   & sensor\_msgs::LaserScan & ROS message type of topic                  \\ 
dataField & scan\_time              & Desired data member of message             \\
callOne    & true                    & Oldest, not yet fetched message is fetched \\
callOne    & false                   & Newest available message is fetched        \\
queueSize  & 1000                    & Size of buffer; queueSize=1000 if omitted  \\ 
useEerosSystemTime  & fals           & Use system time or timestamp of received msg  \\ \bottomrule
\end{tabular}
\end{table}

In table \ref{tableImplementedMsgTypes} are all currently implemented message types and associated data fields.
If your desired message type is not implemented yet, you can easily implement it yourself.
See chapter \ref{sectionImplementMsgType} for a guide to implement additional message types and data fields in \textit{ros-eeros}.

\begin{table}[]
\centering
\caption{Currently implemented message types in ros-eeros}
\label{tableImplementedMsgTypes}
\begin{tabular}{lll}
\cline{2-3}
HAL type  & msgType                    & dataField        \\ \cline{2-3} 
AnalogIn  & std\_msgs::Float64         & -                \\
          & sensor\_msgs::LaserScan    & angle\_min       \\
          &                            & angle\_max       \\
          &                            & angle\_increment \\
          &                            & time\_increment  \\
          &                            & scan\_time       \\
          &                            & range\_min       \\
          &                            & range\_max       \\
AnalogOut & std\_msgs::Float64         & -                \\
          & sensor\_msgs::LaserScan    & angle\_min       \\
          &                            & angle\_max       \\
          &                            & angle\_increment \\
          &                            & time\_increment  \\
          &                            & scan\_time       \\
          &                            & range\_min       \\
          &                            & range\_max       \\
DigIn     & sensor\_msgs::BatteryState & present          \\
DigOut    & sensor\_msgs::BatteryState & present          \\ \cline{2-3} 
\end{tabular}
\end{table}

You can find a complete example, including a *.json file, in the eeros framework (/examples/hal/Ros*).


\subsection{Implementation}
Refere to the documentation of the EEROS HAL\footnote{http://wiki.eeros.org/eeros\_architecture/hal/start?s[]=hal} and check the example in the eeros framework (/examples/ros).

First initialize the HAL in your main function:
\lstset{language=c}
\begin{lstlisting}
...
int main(int argc, char **argv) {
	...
	// HAL
	// ////////////////////////////////////////////////////////////////////////
	HAL& hal = HAL::instance();
	hal.readConfigFromFile(&argc, argv);
	...
\end{lstlisting}

Add the header file to your control system:
\lstset{language=c}
\begin{lstlisting}
#include <eeros/hal/HAL.hpp>
\end{lstlisting}

You can now declare \textit{PeripheralInputs} and \textit{PeripheralOutputs}:
\lstset{language=c}
\begin{lstlisting}
	PeripheralInput<double>		analogIn0;
	PeripheralInput<bool>		digitalIn0;
	PeripheralOutput<double>	analogOut0;
	PeripheralOutput<bool>		digitalOut0;
\end{lstlisting}

Call the constructor of the peripheral IOs with the \textit{signalID} used in the *.json file
\begin{lstlisting}
class MyControlSystem {
public:
	MyControlSystem(double ts, ros::NodeHandle& rosNodeHandler):
		dt(ts),
		...
		analogIn0("scanTimeIn0"),		// argument has to match signalId of json
		digitalIn0("batteryPresent0"),
		analogOut0("scanTimeEchoOut0"),
		digitalOut0("batteryPresentEchoOut0"),
		...
}
\end{lstlisting}


\subsection{Add new ROS message type to the HAL}
\label{sectionImplementMsgType}
\textbf{Preparations}

First you need to check out the master branch of the wrapper library\footnote{https://github.com/eeros-project/ros-eeros}.
After you have implemented and tested your additions, don't hesitate to push your changes to master.


\textbf{Add an input}

In this example describes how to add a new ROS message type to \textit{AnalogIn}.
The procedure to create a new msg type and data field for a \textit{DigIn} is similar.

\textbf{\textit{AnalogIn.hpp:}}
\begin{enumerate}[\hspace{0.5cm}1{.)}]
	\item Include ROS message type
	\lstset{language=c}
	\begin{lstlisting}
...
#include <sensor_msgs/LaserScan.h>
...
	\end{lstlisting}
	
	\item Create new callback functions for ROS
	\lstset{language=c}
	\begin{lstlisting}
...
void sensorMsgsLaserScanAngleMin
	(const sensor_msgs::LaserScan::Type& msg) {
		data = msg.angle_min;
		setTimeStamp(msg.header);
		} ;
...
\end{lstlisting}
\end{enumerate}

\textbf{\textit{AnalogIn.cpp:}}
\begin{enumerate}[\hspace{0.5cm}1{.)}]
	\setcounter{enumi}{2}
	\item Extend parser by selecting a correct callback function for ros
	\lstset{language=c}
	\begin{lstlisting}
	...
	else if ( msgType == "sensor_msgs::LaserScan" ) {
		if 		  ( dataField == "angle_min" )
		 			subscriber = rosNodeHandle->subscribe(topic, queueSize, &AnalogIn::sensorMsgsLaserScanAngleMin, this);
		else if ( dataField == "angle_max" )
					subscriber = rosNodeHandle->subscribe(topic, queueSize, &AnalogIn::sensorMsgsLaserScanAngleMax, this);
		...
	\end{lstlisting}
\end{enumerate}


\textbf{Add an output}

In this example describes how to add a new ROS message type to \textit{AnalogOut}.
The procedure to create a new msg type and data field for a \textit{DigOut} is similar.

\textbf{\textit{AnalogOut.hpp:}}
\begin{enumerate}[\hspace{0.5cm}1{.)}]
	\item Include ROS message type
	\lstset{language=c}
	\begin{lstlisting}
...
#include <sensor_msgs/LaserScan.h>
...
	\end{lstlisting}
	
	\item Declare a set function for ROS
	\lstset{language=c}
	\begin{lstlisting}
...
static void sensorMsgsLaserScanAngleMin (const double value, const ros::Publisher& publisher);
...
	\end{lstlisting}
\end{enumerate}

\textbf{\textit{AnalogOut.cpp:}}
\begin{enumerate}[\hspace{0.5cm}1{.)}]
	\setcounter{enumi}{2}
	\item Extend parser by setting a callback function
	\lstset{language=c}
	\begin{lstlisting}
...
else if ( msgType == "sensor_msgs::LaserScan" ) {
	publisher = rosNodeHandle->advertise<sensor_msgs::LaserScan>(topic, queueSize);
	if ( dataField == "angle_min" )
		etFunction = &sensorMsgsLaserScanAngleMin;
}
...
	\end{lstlisting}
	
	\item Define a set function for ROS
	\lstset{language=c}
	\begin{lstlisting}
void AnalogOut::sensorMsgsLaserScanAngleMin(const double value, const uint64_t timestamp, const ros::Publisher& publisher)
{
	sensor_msgs::LaserScan msg;
	msg.header.stamp = eeros::control::rosTools::convertToRosTime(timestamp);
	msg.angle_min = value;
	publisher.publish(msg);
}
	\end{lstlisting}
\end{enumerate}


\section{EEROS with Gazebo simulation}
\label{EerosWithGazebo}
\subsection{Introduction}
