\chapter{Performance Tests}
\label{chapter:performanceTests}
\section{Einleitung}
%TODO beschreibung des systems (laptop)
In einem Echtzeitsystem wie EEROS ist es besonders wichtig, dass die echtzeitfähigen EEROS Tasks nicht zu fest ausgebremst werden.
Wenn eine ROS-Funktion einen EEROS \textit{Thread} zu lange blockiert, könnte die Echtzeitfähigkeit von EEROS beeinträchtigt werden.

Das ROS-Netzwerk verbindet unterschiedliche Knoten über ein Netzwerk.
Dabei kann es vorkommen, dass einige Knoten viel schneller auf ein \textit{Topic} schreiben, als dass davon gelesen wird.
Auch der gegenteilige Fall, es wird viel schneller gelesen als veröffentlicht wird, kann vorkommen.
Die Software muss in beiden Fällen zuverlässig funktionieren.

Bei vielen eingebetteten Systemen sind Ressourcen wie Prozessorleistung, Arbeitsspeicher und die Bandbreite in einem Netzwerk begrenzt.
Deshalb ist es wichtig abschätzten zu können, wieviel Ressourcen von der Software belegt werden.

Alle oben genannten Eigenschaften werden im folgenden Kapitel gemessen, getestet und beurteilt.

%TODO 100 usec auf topic pushen

\section{Zeitverbrauch von einigen ROScpp Befehlen}
\textbf{Repositories:} \\
\begin{tabular}
  { l						| l			 							l										 l								}
% Name						Repo   									Branch	     							Tag
  SimpleROSNode\_pt1.0		& \textit{Repository}: SimpleROSNode	& \textit{Branch}: performanceTest01	& \textit{Hash}: 3a04f7d 		\\
  EEROSTestApp\_pt1.0		& \textit{Repository}: EEROSTestApp		& \textit{Branch}: performanceTest01	& \textit{Hash}: 745cf77 		\\
\end{tabular}

\subsection{Logging Befehle von ROS}
%TODO ros logging erklären

\subsubsection{Durchführung}
Die Messungen haben gezeigt, dass der erste Aufruf von \textit{ROS\_INFO\_STREAM} viel mehr Zeit braucht als alle darauf folgenden.
Vermutlich werden beim ersten Aufruf alle benötigten Instruktionen in den Cache geladen.
Bei jedem weiteren Aufruf müssen die Instruktionen nicht mehr in den Cache geladen werden.
Aus diesem Grund brauchen alle direkt nachfolgenden Aufrufe weniger Zeit.

Der oben genannte Effekt konnte reproduziert werden indem die Applikation neu gestartet wurde.
Wenn der Prozess für eine Sekunde schlafen gelegt wurde, konnte ein ähnlicher Effekt erzielt werden.

Die Befehle \textit{ROS\_INFO\_STREAM} und \textit{ROS\_INFO} brauchen etwa gleichviel Zeit.

Die Logging-Funktion von EEROS blockiert etwas weniger lang.
Aber auch beim EEROS eigenen Logger tritt das oben genannte Phänomen auf.

\subsubsection{Fazit}
\begin{itemize}
\item Ein Logging-Befehl von ROS braucht ca. \unit[50]{usec} bis \unit[80]{usec}.
\item Alle unmittelbar darauffolgenden Logging-Befehle brauchen ca. \unit[10]{usec} bis \unit[15]{usec}.
\item Ein Logging-Befehl von EEROS braucht ca. \unit[30]{usec} bis \unit[70]{usec}.
\item Alle unmittelbar darauffolgenden Logging-Befehle von EEROS brauchen ca. \unit[3]{usec} bis \unit[15]{usec}.
\item Je nach Anwendung können diese Zeiten in einem Echtzeit Task akzeptabel oder kritisch sein.
\end{itemize}


%TODO anhang


%\section{Tests für ''Essential Features''}
%\subsection{Unabhängiger ROS-Knoten}
%\subsubsection{Zu erfüllende Testbedingungen}
%Eine C++-Applikation schreiben, die folgende Eigenschaften erfüllt:
%\begin{enumerate}
%\item Applikation meldet sich als ROS-Knoten an.
%\item Applikation schickt eine \textit{ROS Log Statement}.
%\end{enumerate}
%
%\subsubsection{Testdurchführung}
%\textbf{Repositories:} \\
%\begin{tabular}
%  { l						| l			 							l								 l								}
%% Name						Repo   									Branch	     					Tag
%  testAppVT2\_t1.0			& \textit{Repository}: eeros-framework	& \textit{Branch}: master		& \textit{Tag}: Test001.0 		\\
%  %TODO 
%\end{tabular}
%
%\textbf{Ablauf: }
%\begin{enumerate}
%\item Den ROS Core mit \textit{\$ roscore} starten.
%\item \textit{\$ rqt\_console} starten.
%\item Mit dem Befehl \textit{\$ rosnode list} 
%\item Testapplikation ''\textit{testAppVT2\_t1.0}'' starten.
%\item Solange die Applikation läuft, muss bei \textit{\$ rosnode list} ein neuer Node aufgelistet sein. \\
%\textbf{Ergebnis:} \checkmark
%\item Bei der \textit{rqt\_console} ist mindestens eine neue \textit{Log-Message} von der Testapplikation erschienen. \\
%\textbf{Ergebnis:} \checkmark
%\item Nachdem die Testapplikation beendet wurde, ist der Knoten der Testapplikation unter \textit{\$ rosnode list} wieder verschwunden. \\
%\textbf{Ergebnis:} \checkmark
%\end{enumerate}
%
%
%\subsection{CMAKE}
%\subsubsection{Zu erfüllende Testbedingungen}
%Eine Klasse in EEROS erstellen, die ROS verwendet und den EEROS Quellcode umschreiben, damit folgende Bedingungen erfüllt werden:
%\begin{itemize}
%\item Wenn ROS installiert ist, wird die neu geschriebene Klasse kompiliert und gegen die entsprechenden ROS-Bibliotheken gelinkt.
%\item Wenn ROS nicht installiert ist, dann wird die neu geschriebene Klasse nicht kompiliert und die restlichen Teile von EEROS kompilieren fehlerfrei.
%\end{itemize}
%
%\subsubsection{Testdurchführung}
%\textbf{Repositories:} \\
%\begin{tabular}
%  { l						| l			 							l								 l								}
%
%% Name						Repo   									Branch Aufwand     				Tag
%  EEROS\_t2.0				& \textit{Repository}: eeros-framework	& \textit{Branch}: ROSVt2		& \textit{Hash}: 8f8d9da		\\
%  EEROS-Applikation\_t2.0	& \textit{Repository}: testAppEEROSVt2	& \textit{Branch}: master		& \textit{Tag}: Test002.0 		\\
%\end{tabular}
%
%\textbf{Ablauf: } 
%\begin{enumerate}
%\item Den \textit{build} Ordner und den \textit{install} Ordner von ''\textit{EEROS\_t2.0} löschen.
%\item \textit{CMAKE} ausführen, \textbf{ohne} dass vorher das \textit{Setup-Skript} von ROS ausgeführt wurde.
%\item Wenn \textit{CMAKE} ausgeführt wird, erscheint unter anderem folgende Ausgabe: \\
%\textit{-- looking for package 'ROS' \\
%-- -> ROS NOT found} \\
%\textbf{Ergebnis:} \checkmark
%\item EEROS baut fehlerfrei und wird richtig installiert. \\
%\textbf{Ergebnis:} \checkmark
%\item Die EEROS-Testapplikation ''\textit{EEROS-Applikation\_t2.0}'' lässt sich \textbf{nicht} bauen, da ein \textit{Header file} von ROS fehlt. \\
%\textbf{Ergebnis:} \checkmark
%\item Den \textit{build} Ordner und den \textit{install} Ordner von ''\textit{EEROS\_t2.0} löschen.
%\item \textit{CMAKE} ausführen, \textbf{nachdem} das \textit{Setup-Skript} von ROS ausgeführt wurde.
%\item Wenn \textit{CMAKE} ausgeführt wird, erscheint unter anderem folgende Ausgabe: \\
%\textit{-- looking for package 'ROS' \\
%-- -> ROS found} \\
%\textbf{Ergebnis:} \checkmark
%\item EEROS baut fehlerfrei und wird richtig installiert. \\
%\textbf{Ergebnis:} \checkmark
%\item Die EEROS-Testapplikation ''\textit{EEROS-Applikation\_t2.0}'' lässt sich bauen. \\
%\textbf{Ergebnis:} \checkmark
%\item Den ROS Core mit \textit{\$ roscore} starten.
%\item \textit{\$ rqt\_console} starten.
%\item Die EEROS-Testapplikation lässt sich mit \textit{\$ sudo -E ./testappEEROSVT2 } starten. \\
%\textbf{Ergebnis:} \checkmark
%\item Bei der \textit{rqt\_console} ist mindestens eine neue \textit{Log-Message} von der Testapplikation erschienen. \\
%\textbf{Ergebnis:} \checkmark
%\end{enumerate}
%
%
%
%\section{Fernsteuerung}
%\subsection{Einfacher ROS-Knoten}
%\subsubsection{Zu erfüllende Testbedingungen}
%In EEROS einen Block für das \textit{Control System} erstellen, welcher das \textit{Topic} vom \textit{turtle\_teleop\_key} einlesen kann.
%\begin{itemize}
%\item Eine EEROS-Testapplikation verwendet einen dafür vorgesehenen Block von EEROS, um die vom \textit{turtle\_teleop\_key} publizierten \textit{Messages} anzuzeigen.
%\end{itemize}
%
%\subsubsection{Testdurchführung}
%\textbf{Repositories:} \\
%\begin{tabular}
%  { l						| l			 							l								 l								}
%
%% Name						Repo   									Branch Aufwand     				Tag
%  EEROS\_t3.0				& \textit{Repository}: eeros-framework	& \textit{Branch}: ROSVt2		& \textit{Hash}: 5bf16d6		\\
%  EEROS-Applikation\_t3.0	& \textit{Repository}: testAppEEROSVt2	& \textit{Branch}: master		& \textit{Tag}: Test003.0		\\
%\end{tabular}
%
%\textbf{Ablauf: } 
%\begin{enumerate}
%\item Den ROS Core mit \textit{\$ roscore} starten.
%\item Testapplikation ''\textit{EEROS-Applikation\_t3.0}'' in einem neuen Terminal starten.
%\item Den \textit{Turtlesim} Knoten mit \textit{\$ rosrun turtlesim turtle\_teleop\_key} starten.
%\item Für die vier Pfeiltaste muss beim Terminal von der Testapplikation eine entsprechende Ausgabe erscheinen. \\
%\textbf{Ergebnis:} \checkmark
%\item Beide Applikationen beenden.
%\item Den \textit{Turtlesim} Knoten mit \textit{\$ rosrun turtlesim turtle\_teleop\_key} starten.
%\item Testapplikation ''\textit{EEROS-Applikation\_t3.0}'' in einem neuen Terminal starten.
%\item Das Teminal mit dem \textit{Turtlesim} Knoten anwählen.
%\item Für die vier Pfeiltaste muss beim Terminal von der Testapplikation eine entsprechende Ausgabe erscheinen. \\
%\textbf{Ergebnis:} \checkmark
%\item Den \textit{Turtlesim} Knoten beenden.
%\item Den \textit{Turtlesim} Knoten mit \textit{\$ rosrun turtlesim turtle\_teleop\_key} neu starten.
%\item Für die vier Pfeiltaste muss beim Terminal von der Testapplikation eine entsprechende Ausgabe erscheinen. \\
%\textbf{Ergebnis:} \checkmark
%\end{enumerate}
%
%
%\subsection{Einfacher ROS-Knoten}
%\subsubsection{Zu erfüllende Testbedingungen}
%In EEROS einen Block für das \textit{Control System} erstellen, welcher von einer EEROS-Applikation benutzt werden kann, um eine beliebige \textit{ROS Message} von einem beliebigen \textit{ROS Topic} lesen zu können.
%Eine EEROS-Testapplikation soll alle \textit{Messages} ausgeben, welche auf den Testknoten veröffentlicht werden.
%%\begin{itemize}
%%\item Ein
%%\end{itemize}
%
%\subsubsection{Testdurchführung}
%\textbf{Repositories:} \\
%\begin{tabular}
%  { l						| l			 							l								 l								}
%
%% Name						Repo   									Branch Aufwand     				Tag
%  EEROS\_t4.0				& \textit{Repository}: eeros-framework	& \textit{Branch}: ROSVt2		& \textit{Hash}: 6de4bdb 		\\
%  EEROS-Applikation\_t4.0	& \textit{Repository}: testAppEEROSVt2	& \textit{Branch}: master		& \textit{Tag}: Test004.0 		\\
%  simpleROSNode\_t4.0		& \textit{Repository}: simpleROSNode	& \textit{Branch}: master		& \textit{Tag}: Test004.0 		\\
%\end{tabular}
%
%\textbf{Ablauf: }
%\begin{enumerate}
%\item Die \textit{EEROS-Applikation\_t4.0} starten.
%\item Ein Testprogramm starten, welches drei \textit{Topics} mit den Namen ''TestTopic1'', ''TestTopic2'' und ''TestTopic3'' erzeugt.
%\item Mit \textit{rqt} und dem Plugin \textit{Message Plugin} \textit{Messages} mit folgende Typen an entsprechende \textit{Topics} senden:
%  \begin{itemize}
%  \item TestTopic1:	std\_msgs/Float64
%  \item TestTopic2:	sensor\_msgs/Joy Message
%  \item TestTopic3:	sensor\_msgs/LaserScan Message
%  \end{itemize}
%\item Die \textit{EEROS-Applikation\_t4.0} gibt korrekt die \textit{Message} aus, welche sie vom \textit{TestTopic1} empfängt. \\
%\textbf{Ergebnis:} \checkmark
%\item Die \textit{EEROS-Applikation\_t4.0} gibt korrekt die \textit{Message} aus, welche sie vom \textit{TestTopic2} empfängt. \\
%\textbf{Ergebnis:} \checkmark
%\item Die \textit{EEROS-Applikation\_t4.0} gibt korrekt die \textit{Message} aus, welche sie vom \textit{TestTopic3} empfängt. \\
%\textbf{Ergebnis:} Übersprungen. Kein Mehrwert zum vorherigen Test.
%\end{enumerate}






