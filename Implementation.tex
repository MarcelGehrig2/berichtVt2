\chapter{Implementation aus EEROS-Entwickler Sicht}


\section{ROSBlock.hpp}
%TODO
Es liegt nahe, in EEROS einen ROS-Block als vererbbare Basis-Klasse zu erstellen.
So könnten alle notwendigen Initialisierungen in einer Basis-Klasse versteckt werden, ohne das sich ein Applikations-Entwickler damit beschäftigen muss.

Dafür ist es notwendig, dass die \textit{Callback function} bereits in der Basisklasse deklariert wird.
Der \textit{Callback function} muss dabei eine Referenz auf die zu empfangende \textit{Message} als Funktionsparameter übergeben werden.
Das folgende Beispiel zeigt eine mögliche Implementation:

\lstset{language=C++}
\begin{lstlisting}
virtual void rosCallbackFct(const sensor_msgs::Joy& msg)
\end{lstlisting}

Natürlich kann eine solche Implementation nur für einen einzigen Typen einer ROS-Nachricht genutzt werden.
Da es aber, wenn man die benutzerdefinierten Nachrichten dazu zählt, eine unbegrenzte Anzahl verschiedener Nachrichtentypen gibt, ist diese Lösung nicht brauchbar. %TODO satzbau

Es liegt nahe den ROS-Block als \textit{Template} zu implementieren.
ROS stellt eine Funktion zu Verfügung, welche den Typ einer ROS-Nachricht zurückgibt.
So könnte man einen benutzerdefinierten Block, abgeleitet vom ROS-Block, mit Hilfe vom Typ der ROS-Nachricht deklarieren:

\lstset{language=C++}
\begin{lstlisting}
ROSBlock<sensor_msgs::Joy::Type> myRosBlock;
\end{lstlisting}

Da der von der ROS-Methode zurückgegebene Typ aber ein abstrakter Typ ist, kann der Block nicht mit dieser Methode deklariert werden.
Versucht man es trotzdem, erhält man folgenden Fehler:

\textit{
error: cannot declare field 'testapp::TestAppCS::rosBlockA' to be of abstract type 'eeros::control::ROSBlock
<sensor\_msgs::Joy\_<std::allocator<void> > >'\\
\-\hspace{2cm} ROSBlock<sensor\_msgs::Joy> rosBlockA;
}
